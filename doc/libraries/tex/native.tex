\documentclass[pdflatex,11pt,letter]{article}

\begin{document}

\begin{center}
Version 1.0\hfill{}September 30, 2013\\[0.5cm]
{\Huge Creating Native Libraries for ProcessJ Using C}\\[0.5cm]
by Matt B. Pedersen
\end{center}

\section{Introduction}

A {\it native} library in ProcessJ is a library that is written in C. There are 
two different kinds of native libraries in ProcessJ:

\begin{enumerate}
\item A library that maps {\bf directly} to an existing C library like for example {\tt math.h}.
\item A library that is implemented in C but does not map to any existing libraries.
\end{enumerate}

\noindent
All ProcessJ files that implement libraries must include the pragma
{\tt LIBRARY} as well as {\tt FILE} and either {\tt NATIVELIB} (if
mapping to an existing library) or {\tt NATIVE} if the implementation
will be written in C.

\noindent
All ProcessJ library files must declare a package name.

\section{Mapping to and Existing Library}

To define a native library mapping in ProcessJ follow the following example where 
we map some of the existing constants an procedures from C's {\tt math.h} library to a 
library called {\tt math} in the {\tt std} package. Simply create this ProcessJ file:

\begin{verbatim}
#pragma LIBRARY;
#pragma NATIVELIB "math.h";
#pragma LANGUAGE "C";
#pragma FILE "math";

package std;

public native const double M_PI;        /* pi */
public native const double M_PI_2;      /* pi/2 */
public native const double M_PI_4;      /* pi/4 */
public native const double M_1_PI;      /* 1/pi */ 
public native const double M_2_PI;      /* 2/pi */ 
public native const double M_2_SQRTPI;  /* 2/sqrt(pi) */
public native const double M_SQRT2;     /* sqrt(2) */
public native const double M_SQRT1_2;   /* 1/sqrt(2) */

public native proc double acos(double x) ;
public native proc double asin(double x) ;
public native proc double atan(double x) ;
public native proc double atan2(double y, double x) ;
public native proc double cos(double x) ;
public native proc double cosh(double x) ;
public native proc double sin(double x) ;
public native proc double sinh(double x) ;
public native proc double tanh(double x) ;
public native proc double exp(double x) ;
public native proc double ldexp(double x, int exponent) ;
public native proc double log(double x) ;
public native proc double log10(double x) ;
public native proc double pow(double x, double y) ;
public native proc double sqrt(double x) ;
public native proc double ceil(double x) ;
public native proc double fabs(double x) ;
public native proc long abs(long x) ;
public native proc int abs(int x) ;
public native proc double floor(double x) ;
public native proc double fmod(double x, double y) ;
\end{verbatim}

\noindent
Both constants and procedures must be declared native and procedures cannot have a body
and constants cannot be initialized.\\

\noindent
The pragmas instruct the compiler to create a library with native C mappings to {\tt $<$math.h$>$}.\\

\noindent
The compiler will generate: {\tt std\_math.h} which will look like this:

\begin{verbatim}
#ifndef _LIB_STD_AMTH_
#define _LIB_STD_MATH_
#include <math.h>
#endif
\end{verbatim}

\noindent
which should be moved to {\tt lib/C/include}\\

\noindent
and {\tt std\_math.c} which will look like this:

\begin{verbatim}
#ifndef _STD_MATH_H
#define _STD_MATH_H
#include "std_math.h"
#endif
\end{verbatim}

\noindent
which should be moved to {\tt lib/C/src}.\\

\noindent
The object file {\tt std\_math.o} should be moved to {\tt /lib/C/obj}.\\

\noindent
The original ProcessJ file, {\tt math.pj} should be moved to {\tt include/C/std/}.\\

\noindent
To avoid manually moving files around you can use the {\tt pjc-install-c-library} script.

\section{Self Written C libraries}

To write a native library in C that does not map to an existing C
library the {\tt NATIVELIB} pragma should be replaced by the {\tt
  NATIVE} pragma.\\

\noindent
Procedures should use he {\tt native} keyword and have no
implementation in the ProcessJ file.\\

\noindent
Constants should {\bf not} be declared {\tt native}.
Non-native constants are declared without the
{\tt native} keyword and an initializer is required.\\

\noindent
ProcessJ allows procedure overloading, C does not, we will return to
this issue a little later.\\

\noindent
Here is an example of a ProcesJ native library that does not map
directly to an existing C library:

\begin{verbatim}
#pragma LIBRARY;
#pragma NATIVE;
#pragma FILE "file";
#pragma LANGUAGE "C";

package io;

public native const string READ;
public native const string WRITE;

public native proc int fileOpen(string fileName, string mode) ;
public native proc int fileClose(int file) ;

public native proc int fileWrite(int file, string data) ;
public native proc int fileWrite(int file, int data) ;
public native proc int fileWrite(int file, float data) ;
public native proc int fileWrite(int file, long data) ;
public native proc int fileWrite(int file, double data) ;
public native proc int fileWrite(int file, short data) ;
\end{verbatim}

\noindent
The ProcessJ compiler will generate a header ({\tt .h}) file {\tt
  io\_file} which will look like this:


\begin{verbatim}
#ifndef _LIB_IO_FILE_
#define _LIB_IO_FILE_

// Add #include statements and constants here

int io_fileOpen_TT(char* fileName, char* Mode) ;
int io_fileClose_I(int file) ;
int io_fileWrite_IT(int file, char* data) ;
int io_fileWrite_II(int file, int data) ;
int io_fileWrite_IF(int file, float data) ;
int io_fileWrite_IJ(int file, long data) ;
int io_fileWrite_ID(int file, double data) ;
int io_fileWrite_IS(int file, short data) ;

#endif
\end{verbatim}

\noindent
Since the ProcessJ file contained multiple definitions (with different
signatures) of the {\tt fileWrite} procedure, these must be generated
with different names for the C compiler to be happy. Since the
signatures of the parameters will differ, we use these to distinguish
the different versions of the procedure.\\

\noindent
The generated C file looks like this:

\begin{verbatim}
#ifndef _IO_FILE_H
#define _IO_FILE_H
#include "io_file.h"
int io_fileOpen_TT(char* fileName, char* Mode) {
  // implementation code goes here.
}

int io_fileClose_I(int file) {
  // implementation code goes here.
}

int io_fileWrite_IT(int file, char* data) {
  // implementation code goes here.
}

int io_fileWrite_II(int file, int data) {
  // implementation code goes here.
}

int io_fileWrite_IF(int file, float data) {
  // implementation code goes here.
}

int io_fileWrite_IJ(int file, long data) {
  // implementation code goes here.
}

int io_fileWrite_ID(int file, double data) {
  // implementation code goes here.
}

int io_fileWrite_IS(int file, short data) {
  // implementation code goes here.
}

#endif
\end{verbatim}

\noindent
Also note, that the ProcessJ {\tt string} type is converted to {\tt
  char*} and the {\tt boolean} type is converted to {\tt int}.\\

\noindent
Implement the library in the C templates generated by the compiler; if additional 
internal functionality is required it can be added in this file as well, but be
careful with naming these functions and ``global'' variables.\\

\noindent
The location of the {\tt .h}, {\tt .c}, {\tt .o} and {\tt .pj} files 
go in the same locations as mentioned in the previous section.\\

\noindent
The {\tt pjc-install-c-library} can be used in the exact same way as
in the previous section: if the third parameters is {\tt test} the
script will check for all the files and compile the library, but not
move the files. To force a move of the files, replace {\tt test} by {\tt install}.








\end{document}
